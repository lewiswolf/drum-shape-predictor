% ARTICLE PREAMBLE
\title{Predicting the Shape of Arbitrary 2-dimensional Drums}
\author{Lewis Wolstanholme}
\date{\getDate\today}
\maketitle

\noindent For this project, random convex shapes were generated according to the efficient algorithm presented by \citet{valtrProbabilityThatRandom1995}.
Concave shapes were individually generated using a set of \(n\) random cartesian products, centred around the origin, connected to their closest neighbours according to \(\displaystyle \arctan \left( \frac{y}{x} \right)\).
The dataset itself was formed around a Bernoulli distribution of these two algorithms.
Given that the latter algorithm, used to create concave shapes, has a measurable probability of also returning a convex shape~\citep{valtrProbabilityThatRandom1995}, we can assume that the distribution of \(n\) sided convex drums will be equivalent to
\[ p_n = \frac{1}{2} + \frac{1}{2}\left( \frac{\genfrac(){0pt}{1}{2n - 2}{n-1}}{n!}\right)^2\]
where \(n\) is determined by a uniform distribution ranging from 3 to \_.

For the physical model itself, parameters were defined according to \_. The variable \(\lambda\) is used to represent the Courant number~\citep{courantPartialDifferenceEquations1967}.